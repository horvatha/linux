\documentclass[a4paper]{article}
\usepackage[top=1in, bottom=1.25in, left=1.05in, right=1.05in]{geometry}
% 12 (10, 11) pontos betű: \documentclass[a4paper,12pt]{article}

%%%%%%%%%%%%%%%%%%%%%%%%%%%%%%%%%%%%%%%%%%%%%%%%%%
%%%% PREAMBULUM= A \begin{document}-ig tartó rész
%%%%%%%%%%%%%%%%%%%%%%%%%%%%%%%%%%%%%%%%%%%%%%%%%%

%%%%%%%%%%%%%%%%%%%%%%%%%%%%%%%%%%%%%%%%%%%%%%%%%%
%%%% A KARAKTEREK KÓDOLÁSÁVAL kapcsolatos csomagok
% KÓDOLÁSRA FIGYELNI, ALÁBB BEÁLLÍTANI !!! latin2 vagy utf-8 ?

% iso-8859-2 azaz latin-2 kódolás esetén ez a rész kell:
%\usepackage{t1enc}
%\usepackage[latin2]{inputenc}

% (utf-8-ra kódolhatjuk az iso-8859-est
% az enconv paranccsal [enca csomag])
% utf-8 kódolás esetén ez a rész kell:

\usepackage{ucs}
\usepackage{html}
\usepackage[T1]{fontenc}
\usepackage[utf8x]{inputenc}
\usepackage{tikz}
\tikzset{ %inner sep = 0.5mm,
  >=latex,
  {entity/.style} ={draw,rectangle, inner sep=3pt},
  {property/.style} ={},
  {primaryKey/.style} ={property,font=\bf},
 }

%%%%%%%%%%%%%%%%%%%%%%%%%%%%%%%%%%%%%%%%%%%%%%%%%%
%%%% A MAGYAR NYELVVEL kapcsolatos csomagok
\usepackage[magyar]{babel}
\usepackage{indentfirst}  % Az első bekezdést is behúzza.

\frenchspacing   % A mondatvégek után azonos szóköz van, mint máshol
 % (Az angolban nagyobb a szokás. Ez az alapbeállítás.)

% A pdf kisebb és olvashatóbb lesz, ha times betűket használok sima
% (nem pdf-) latex esetén.
% \usepackage{times}

%%%%%%%%%%%%%%%%%%%%%%%%%%%%%%%%%%%
%%%% Az ÁBRÁKHOZ hasznos csomagok

%\usepackage{graphics} % ábrák beillesztésének bővebb paraméterezése
% \usepackage{psfrag} % PostScript ábrák feliratainak cserélése TeX-esre

%%%%%%%%%%%%%%%%%%%%%%%%%%%%%%%%%%%%%%%%%%%%%%
%%%% MATEKKAL KAPCSOLATOS CSOMAGOK, DEFINÍCIÓK

% Az Amerikai Matematikai Társulat (AMS)
% hasznos csomagja  (pl. \dfrac)
%% \usepackage{amsmath}

% Magyar függvénynevek
%% \DeclareMathOperator{\tg}{tg}
%% \DeclareMathOperator{\sh}{sh}
%% \DeclareMathOperator{\ch}{ch}
%% \DeclareMathOperator{\cth}{cth}
%% \DeclareMathOperator{\ctg}{ctg}
%% \DeclareMathOperator{\arctg}{arctg}
%% \DeclareMathOperator{\arcctg}{arcctg}
%% \DeclareMathOperator{\arsh}{arsh}
%% \DeclareMathOperator{\arch}{arch}
%% \DeclareMathOperator{\arth}{arth}
%% \DeclareMathOperator{\arcth}{arcth}

% Ezzel a duplaszárú betűk elérhetőek.
% Pl. valós számok halmazjele:  \mathbb{R}
% (Nem tökéletes, mert mindkét oldalon dupla.)
%\usepackage{amssymb}

%%%%%%%%%%%%%%%%%%%%%%%%%%%%%%%%%%%%%%%%%%%%%%%%%%%%
%%%% KIMONDOTTAN EHHEZ A FÁJLHOZ DEFINIÁLT PARANCSOK
%\usepackage{html} %html-címekhez hasznos


\begin{document}

\title{Linux alkalmazása feladatok}
\author{Horváth Árpád}
\maketitle

Az alábbi feladatok mind az egyetemhez kapcsolódó django-s feladatok.
Adatmodell, admin oldal, valamint template-tel elkészített oldal kell.
A template-ekkel készített oldal esetén minden egyed listázható legyen,
felsorolva: hányan vagy mik tartoznak hozzá.

Legalább egy idegen kulcs személyenként + több-a-többhöz kapcsolat.

Admin oldalnál valamilyen ,,inline''.

Ékezet nélküli mezőneveket/adatmodell-neveket használjunk, és lehetőleg angolul.

A részt vevő személyek számára lehet alkudni: részletesebb kidolgozás,
több adatmodell (tábla) esetén több személyé lehet.

\section{Feladatok}

\subsection{Meglévő oktatóoldal továbbfejlesztése}

Pillanatnyilag (2017 március) nem elérhető az oldal szerverköltözés
miatt,\\ várható helye \verb+http://pyedu.hu/kurzusok+ és/vagy
\verb+http://kurzusok.pyedu.hu+ .

A két videót érdemes megnézni: pyEdu ill. udacity.

\verb+http://django.amk.uni-obuda.hu/segedletek/pyedu/pyEdu_pycodes.ogv+

\verb+http://arek.uni-obuda.hu/~horvatha/pyEdu_video/pyEdu/udacity_introduction.ogv+

\verb+git clone ssh://git@django.arek.uni-obuda.hu:122/home/git/pyEdu.git+
(jelszó kell hozzá)

A Pythonos programozó feladatokhoz az ACE integrálása a programkód
szerkesztés megkönnyítésére.

\verb+https://ace.c9.io/+

Esetleg ez is hasznos lehet:

\verb+https://pypi.python.org/pypi/django-ace-overlay/0.3+

A unittesztek eredményeinek átrendezése, áttekinthetőbbé tétele.

És egy halom feladat, ami a TODO-kban van. TDK-munkának, szakdolgozatnak
jó.

\subsection{Meglévő esemény-nyilvántartó továbbfejlesztése}

\verb!http://django.arek.uni-obuda.hu/django/elft/naptar/!

\verb!git clone https://github.com/horvatha/elft!
(ez csak az alkalmazás, projekt kell köré)

Megbeszélés szerinti létszámban: versenyek, konferenciák beillesztése.

Mindkettőnél:

A \emph{konferencia} egészére lehessen tárolni a kezdés és befejezés dátumát
kötelezően, az időpontokat esetlegesen.
Lehessen tárolni az egyes előadásokat,
az előadások:
\begin{itemize}
    \item szerzőit,
    \item az előadások is besorolhatóak legyenek kategóriákba,
	Nem kötelezőként:
    \item az előadások kezdőidőpontját, és befejezési
	időpontját. Mentéskor az adatmodell ellenőrizze, hogy ezek az időpontok
	a konferencia időpontján belül vannak-e.
    \item az előadás kivonatát (absztraktját).
\end{itemize}

A \emph{versenyeknél}:
Dátumot, időpontot, mint a konferenciáknál.
Nem kötelezőként
\begin{itemize}
    \item a zsűrielnököt,
    \item zsűritagokat,
    \item résztvevőket lehessen tárolni, résztvevőknél a helyezéseket is.
\end{itemize}

Át lehet dolgozni, hogy az adatmodellek nevei angolul legyenek, de a
honlapon magyarul jelenjenek meg a feliratok, vagy úgy, hogy nyelv
választásától függően az adott nyelven jelenjen meg minden.

\subsection{Énekkaros alapoldal}

Cél olyan oldal kifejlesztése, amely az énekkarok igényeit kis
átalakítással kielégíti.

Főbb szolgáltatások.

\begin{itemize}
    \item repertoár tárolása, keresése szerző, szerző nemzetisége, cím szavai, kor, stílus szerint
    \item események (elnevezés, hely, nap, kezdésidő, műsor) megjelenítése
    \item eseménynaptár megjelenítése,
    \item az egyes művekhez a kotta, felvétel és egyéb fájlok feltöltési
	lehetősége
    \item bemutatkozás
    \item keresés hangfelvételek között
    \item fogalomtár, lexikon\ldots
\end{itemize}

\subsection{Környezetvédelmi nyilvántartó}

A Juglans Alba Mérnöki Iroda által használt adatok tárolása. A
vezetőjének a levele:

\begin{verbatim}
Feladat: 

Termelő vállalatok hulladék nyilvántartásának vezetése, illetve éves
adatszolgáltatási kötelezettség teljesítéséhez segítség nyújtása. A
bevalláshoz csatoltam egy mintát, ez éles adat
így kezeljétek légyszíves bizalmasan, de látjátok, hogy élő példáról van
szó.

Szükséges adatok:

Vállalat adatai
termelő vállalat neve,
termelő címe (székhely)
termelő vállalat KÜJ száma (Környezetvédelmi Ügyfél Jel) 9 számjegyű
vállalat KSH száma
cégvezető neve(i)
telefon/fax/email-cím

Telephelyek adatai
telephelyek ahol a hulladékok keletkeznek (a nyilvántartást mindegyik
telephelyre külön-külön kell megcsinálni)
Telephely címe
Telephely KTJ száma (Környezetvédelmi Területi Jel) mindegyik telephelynek
önálló KTJ száma van, ez is 9 számjegyből álló azonosító

Telephelyeken végzett tevékenységek azonosítása
minden olyan tevékenység, amelyből hulladék keletkezik, egy telephelyen több
különböző tevékenység is folyhat (pl. fémek bevonatolása és gépkarbantartás,
valamint irodai tevékenység is)
a tevékenységeket célszerű TEÁOR számmal azonosítani

Keletkező hulladékok adatai
Nyilvántartás rendszere: Napi nyilvántartás (keletkezés és átadás
kezelőknek)
A keletkezett hulladék
megnevezése (lásd melléklet excel)
azonosító kódja (lásd mellékelt excel 2+2+2 számjegyű azonosító kódok)
hulladék megjelenési formája (S-szilárd, F-folyékony, I-iszap)
Keletkezett mennyiség (kg-ban)
keletkezés dátuma
kezelőnek átadott mennyiség (kg-ban)
Gyűjtőhelyen lévő aktuális mennyiség (kg-ban)

Hulladék átvevők adatai (kezelők)
Hulladékkezelő neve
Székhelye
Kezeléssel érintett telephely címe (ilyen több is lehet a kezelőnek,
mindegyiket külön-külön kell azonosítani, itt a KTJ szám különbözteti meg az
egyes kezelő telephelyeket)
KÜJ száma
KTJ száma, ahol majd a kezelés megtörténik.
Adott hulladékra vonatkozó kezelés kódja (kezelési kódok forrása: 2012. évi
CXXXLV törvény utolsó szakasza és 439/2012 (XII. 29.) Korm. rendelet
mellékletei
Kezelésre átvett hulladék mennyisége (kg) FIGYELEM akár egy hulladék fajtát
is lehet több különböző kezelési kódra meg egy telephelyen belül is
Átadása dátuma
\end{verbatim}

Saját továbbfejlesztési ötletek. Tárolja a telephelyek esetén:
\begin{itemize}
    \item Lekérdezés hulladék szerint.
    \item Hulladékátvevők listája.
    \item \ldots
    \item földrajzi szélességet és hosszúságot (ellenőrizze, az intervallumokat,
	hogy Magyarországon belül legyen), esetleg a PostGIS
	használatával.
\end{itemize}

\subsection{Tananyag-nyilvántartó}

Az egyes tananyagok tartalmazhatnak egy tetszőleges típusú fájlt. Egy
leírást (ez is tetszőleges fájl, de a .pdf, .docx vagy .doc
kiterjesztést ellenőrzi, esetleg csupa nagybetűvel).

A tananyag további jellemzői:
\begin{itemize}
    \item szerző (több-a-többhöz)
    \item kategória (több-a-többhöz)
    \item feltöltés dátuma
\end{itemize}

Minden tananyaghoz rendelhető kategória. Minden kategóriához
szülőkategóriák is rendelhetőek, így tetszőleges mélységű hierarchia
hozható létre. (Ha lehet, mentéskor ellenőrizze, hogy valami nem lesz-e
saját maga ,,leszármazottja'', azaz irányított körmentes gráfot
alkossanak a kategóriák.)

Az oldalon listázni lehessen a legkésőbb feltöltött tananyagokat.
Szerzők és kategóriák szerint kereshetőek legyenek a tananyagok.

A tananyagok feltölthetőek legyenek legalább az admin oldalon.

A tananyagok mellett esetleg lehetnének oktatáshoz hasznos URL-ek
tárolva, amelyekhez az előzőekben már ismertetet kategóriákat szintén
hozzá lehetne rendelni.

\subsection{Könyvtári nyilvántartás}

2 fő

A könyveket, szerzőket és könyvtár-felhasználókat és kölcsönzéseket kell
nyilvántartani.


\begin{verbatim}
Book(author (több-a-többhöz), title, publisher, year, tipus)

Author(first_name, last_name,
       birth_year, birth_place,
       death_year, death_place)

User(first_name, last_name,
     birth_year, birth_place,
     work_place, email, phone)

Lending(user, book, lending_date)

Book.type:
 1m 1 hónapos kölcsönzés
 1w 1 hetes kölcsönzés
 0w nem kölcsönözhető
\end{verbatim}

\subsection{Konzultációs időpontok nyilvántartása}

1 fő

Az intézet oktatói konzultációs időpontokat írhatnak ki, melyekhez
szöveges megjegyzést fűzhetnek. Egy konzultációnak van helye és kezdő és
végső időpontja. A termekhez férőhely tartozik.

\subsection{Garai és műhely-előadások nyilvántartása}

1 fő

A szerdai napokon általában fix időpontban előadások szerepelnek.
Ezeknek van előadójuk, címük, esetleg alcímük és rövid leírásuk. Az
előadónak lehet valamilyen leírása (X cég vagy
főiskolai tanár)

\subsection{TDK/projekt/szakdolgozat témakörök és dolgozatok nyilvántartása}

2 fő (ha a záróvizsga-nyilvántartást (jegy, elnök\ldots) is hozzávesszük, 4 fős is lehet)

Bármely oktató írhat ki témaköröket, melynél megjelöl a három
kategória közül legalább egyet. A hallgatók megtekinthetik szűrve az
egyes típusokra a kínálatot, és láthatják az e-mailcímet valamilyen
robottal nem összegyűjthető formában.

Egy főre az egyik típus elég.
Már kész dolgozatok tárolására és visszakeresésére szolgál.

A szakdolgozatoknak egy szerzője, egy belső konzulense, bírálója és
esetleg külső konzulense van. Lehetnek titkosítottak is, mely esetben
nem érhetőek el a könyvtárban.

A TDK-dolgozatoknál tárolni kell a szerzők mellett az konzulenst is, hogy
főiskolai ill. esetleg országos TDK-n milyen eredményt ért el milyen
szekcióban. A helyezés lehet 1. 2. 3. vagy különdíj. A helyezéshez
megjegyzés fűzhető (milyen különdíj, megosztott x. helyezett...).

Lehessen bárki által lekérdezni.

\subsection{Dolgozók nyilvántartása}

1 fő

    \begin{tikzpicture}[xscale=2]
        \node[entity] (ent1) at (-2.3, 0) {Születési hely};
        \node[entity] (ent2) at ( 0, 0) {Dolgozó};
        \node[entity] (ent3) at ( 2, 0) {Végzettség};
        \draw[<->] (ent1) edge node[near start,above] {1}
            node[near end,above] {N} (ent2);
        \draw[<->] (ent2) edge node[near start,above] {N}
            node[near end,above] {M} (ent3);
            \foreach \type/\property/\x/\entity in {property/HelyNév/-3.3/ent1,
            property/Szélesség/-2.3/ent1, property/Hosszúság/-1.3/ent1,
            property/UtóNév/-.5/ent2, property/CsaládNév/+.3/ent2,
	    property/SzulDatum/1.2/ent2,
            property/Szint/2/ent3, property/Vegzettseg/3/ent3}{
            \node[\type] (\property) at (\x, -1) {\property};
            \draw[<-] (\property) -- (\entity);
        }
    \end{tikzpicture}

    Szint: felsőfokú szakképzés, alap, mester, doktori

\subsection{Óvodai nyilvántartás}

1 fő

    \noindent
    \begin{tikzpicture}[xscale=2]
        \node[entity] (szh) at (-2, 0) {Születési hely};
        \node[entity] (on) at ( 0, 0) {Óvó néni};
        \node[entity] (o) at ( 2, 0) {Óvodás};
        \draw[<->] (szh) edge node[near start,above] {1} node[near end,above] {N} (on);
        \draw[<->] (on) edge node[near start,above] {N} node[near end,above] {M} (o);
        \foreach \type/\property/\x/\entity in {property/TelepulesNev/-3/szh,
                            property/Orszag/-2/szh,
                            property/Nev/-1/on, property/SzuletesiEv/0/on,
                            property/Nev/1/o,
                            property/AnyjaNeve/2/o, property/SzulDatum/3/o}{
            \node[\type] (\property) at (\x, -1) {\property};
            \draw[<-] (\property) -- (\entity);
        }
    \end{tikzpicture}

\subsection{Számla-nyilvántartás}

1 fő

    \begin{tikzpicture}[xscale=2]
        \node[entity] (Ugyfel) at (-2, 0) {Ügyfél};
        \node[entity] (Szamla) at ( 0, 0) {Számla};
        \node[entity] (NaploBejegyzes) at ( 2, 0) {NaplóBejegyzés};
        \draw[<->] (Ugyfel) edge node[near start,above] {N} node[near end,above] {M} (Szamla);
        \draw[<->] (Szamla) edge node[near start,above] {1} node[near end,above] {N} (NaploBejegyzes);
        \foreach \type/\property/\x/\entity in {property/ÜgyfKód/-3/Ugyfel,
                            property/Név/-2.3/Ugyfel,
                            property/Cím/-1.8/Ugyfel,
                            property/Számlaszám/-1/Szamla,
                            property/Egyenleg/0/Szamla,
                            property/Üzenet/1/NaploBejegyzes,
                            property/Időpont/2/NaploBejegyzes,
                            property/PénzMozgás/3/NaploBejegyzes}{
            \node[\type] (\property) at (\x, -1) {\property};
            \draw[<-] (\property) -- (\entity);
        }
    \end{tikzpicture}

\subsection{Csillagda-látogatás foglalása}

1 fő

    \begin{tikzpicture}[xscale=2]
        \node[entity] (ent1) at (-2.3, 0) {Előadók};
        \node[entity] (ent2) at ( 0, 0) {Alkalmak};
        \node[entity] (ent3) at ( 2, 0) {Intézmény};
        \draw[<->] (ent1) edge node[near start,above] {M}
            node[near end,above] {N} (ent2);
        \draw[<->] (ent2) edge node[near start,above] {N}
            node[near end,above] {1} (ent3);
            \foreach \type/\property/\x/\entity in {property/név/-3.3/ent1,
            property/foglalkozás/-2.3/ent1, property/telefonszám/-1.3/ent1,
            property/dátum/-.5/ent2, property/kezdet/+.3/ent2,
	    property/név/1.0/ent3,
            property/kontaktszemély/1.82/ent3, property/telefonszám/3/ent3}{
            \node[\type] (\property) at (\x, -1) {\property};
            \draw[<-] (\property) -- (\entity);
        }
    \end{tikzpicture}

\section{Határidő}
A egyetem moodle oldalán (https://elearnig.uni-obuda.hu) található
követelményrendszer szerint.
\end{document}

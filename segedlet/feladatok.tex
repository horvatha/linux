\documentclass[a4paper]{article}
% 12 (10, 11) pontos betű: \documentclass[a4paper,12pt]{article}

%%%%%%%%%%%%%%%%%%%%%%%%%%%%%%%%%%%%%%%%%%%%%%%%%%
%%%% PREAMBULUM= A \begin{document}-ig tartó rész
%%%%%%%%%%%%%%%%%%%%%%%%%%%%%%%%%%%%%%%%%%%%%%%%%%

%%%%%%%%%%%%%%%%%%%%%%%%%%%%%%%%%%%%%%%%%%%%%%%%%%
%%%% A KARAKTEREK KÓDOLÁSÁVAL kapcsolatos csomagok
% KÓDOLÁSRA FIGYELNI, ALÁBB BEÁLLÍTANI !!! latin2 vagy utf-8 ?

% iso-8859-2 azaz latin-2 kódolás esetén ez a rész kell:
%\usepackage{t1enc}
%\usepackage[latin2]{inputenc}

% (utf-8-ra kódolhatjuk az iso-8859-est
% az enconv paranccsal [enca csomag])
% utf-8 kódolás esetén ez a rész kell:

\usepackage{ucs}
\usepackage{html}
\usepackage[T1]{fontenc}
\usepackage[utf8x]{inputenc}

%%%%%%%%%%%%%%%%%%%%%%%%%%%%%%%%%%%%%%%%%%%%%%%%%%
%%%% A MAGYAR NYELVVEL kapcsolatos csomagok
\usepackage[magyar]{babel}
\usepackage{indentfirst}  % Az első bekezdést is behúzza.

\frenchspacing   % A mondatvégek után azonos szóköz van, mint máshol
 % (Az angolban nagyobb a szokás. Ez az alapbeállítás.)

% A pdf kisebb és olvashatóbb lesz, ha times betűket használok sima
% (nem pdf-) latex esetén.
% \usepackage{times}

%%%%%%%%%%%%%%%%%%%%%%%%%%%%%%%%%%%
%%%% Az ÁBRÁKHOZ hasznos csomagok

%\usepackage{graphics} % ábrák beillesztésének bővebb paraméterezése
% \usepackage{psfrag} % PostScript ábrák feliratainak cserélése TeX-esre

%%%%%%%%%%%%%%%%%%%%%%%%%%%%%%%%%%%%%%%%%%%%%%
%%%% MATEKKAL KAPCSOLATOS CSOMAGOK, DEFINÍCIÓK

% Az Amerikai Matematikai Társulat (AMS)
% hasznos csomagja  (pl. \dfrac)
%% \usepackage{amsmath}

% Magyar függvénynevek
%% \DeclareMathOperator{\tg}{tg}
%% \DeclareMathOperator{\sh}{sh}
%% \DeclareMathOperator{\ch}{ch}
%% \DeclareMathOperator{\cth}{cth}
%% \DeclareMathOperator{\ctg}{ctg}
%% \DeclareMathOperator{\arctg}{arctg}
%% \DeclareMathOperator{\arcctg}{arcctg}
%% \DeclareMathOperator{\arsh}{arsh}
%% \DeclareMathOperator{\arch}{arch}
%% \DeclareMathOperator{\arth}{arth}
%% \DeclareMathOperator{\arcth}{arcth}

% Ezzel a duplaszárú betűk elérhetőek.
% Pl. valós számok halmazjele:  \mathbb{R}
% (Nem tökéletes, mert mindkét oldalon dupla.)
%\usepackage{amssymb}

%%%%%%%%%%%%%%%%%%%%%%%%%%%%%%%%%%%%%%%%%%%%%%%%%%%%
%%%% KIMONDOTTAN EHHEZ A FÁJLHOZ DEFINIÁLT PARANCSOK
%\usepackage{html} %html-címekhez hasznos


\begin{document}

\title{Linux alkalmazása, adatbázis feladatok}
\author{Horváth Árpád}
\maketitle

Az alábbi feladatok mind a főiskolához kapcsolódó adatkezelési
feladatok. A megoldásuk közös adatbázissal lehet hatékony. A feladat
megoldásához a mail.roik.bmf.hu szerver PostgreSQL adatbázisát kell
felhasználni, valamint a felülethez webes oldalt kialakítani a Python
CGI-vel.
Az egyes táblázatokat saját jelszó megadása után módosíthatják egyes
oktatókweben keresztül.
Az oktatók azonosítója a NEPTUN-kód, de a jelszó szabadon
megválasztható.

Az oktatók táblázat létrehozását, a weboldalhoz tartozó kinézetet (css)
és a jogosultság-kezelést közösen kell megoldani.

Egyes lekérdezéseket lehessen LaTeX-en keresztül PDF-ben is letölteni.

\section{Részfeladatok}

\subsection{Konzultációs időpontok nyilvántartása}
Az intézet oktatói konzultációs időpontokat írhatnak ki, melyekhez
szöveges megjegyzést fűzhetnek. Egy konzultációnak van helye és kezdő és
végső időpontja. A hely egyelőre szövegesen megadható (nem kell termek
táblázat).

Ehhez a feladathoz tartozik a többi feladat összefogása is.

\subsection{Műhely előadások nyilvántartása}
A szerdai napokon általában fix időpontban előadások szerepelnek.
Ezeknek van előadójuk, címük, esetleg alcímük és rövid leírásuk. Az
előadónak lehet valamilyen leírása (X cég vagy
főiskolai tanár)

\subsection{TDK/projekt/szakdolgozat témakörök nyilvántartása}
Bármely oktató írhat ki témaköröket, melynél megjelöl a három
kategória közül legalább egyet. A hallgatók megtekinthetik szűrve az
egyes típusokra a kínálatot, és láthatják az e-mailcímet valamilyen
robottal nem összegyűjthető formában.

Módosíthatja a hozzárendelt oktató.

\subsection{TDK/szakdolgozat dolgozatok nyilvántartása}
Egy főre az egyik típus elég.
Már kész dolgozatok tárolására és visszakeresésére szolgál.

A szakdolgozatoknak egy szerzője, egy belső konzulense, bírálója és
esetleg külső konzulense van. Lehetnek titkosítottak is, mely esetben
nem érhetőek el a könyvtárban.

A TDK-dolgozatoknál tárolni kell a szerzők mellett az konzulenst is, hogy
főiskolai ill. esetleg országos TDK-n milyen eredményt ért el milyen
szekcióban. A helyezés lehet 1. 2. 3. vagy különdíj. A helyezéshez
megjegyzés fűzhető (milyen különdíj, megosztott x. helyezett...).

Módosíthatja a hozzárendelt oktató.
Mindkét típust lehessen bárki által lekérdezni.

\section{Határidő}
A BMF-es moodle oldalon (https://elearnig.bmf.hu).
\end{document}

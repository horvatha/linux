\documentclass[a4paper]{article}
% 12 (10, 11) pontos betű: \documentclass[a4paper,12pt]{article}

%%%%%%%%%%%%%%%%%%%%%%%%%%%%%%%%%%%%%%%%%%%%%%%%%%
%%%% PREAMBULUM= A \begin{document}-ig tartó rész
%%%%%%%%%%%%%%%%%%%%%%%%%%%%%%%%%%%%%%%%%%%%%%%%%%

%%%%%%%%%%%%%%%%%%%%%%%%%%%%%%%%%%%%%%%%%%%%%%%%%%
%%%% A KARAKTEREK KÓDOLÁSÁVAL kapcsolatos csomagok
% KÓDOLÁSRA FIGYELNI, ALÁBB BEÁLLÍTANI !!! latin2 vagy utf-8 ?

% iso-8859-2 azaz latin-2 kódolás esetén ez a rész kell:
%\usepackage{t1enc}
%\usepackage[latin2]{inputenc}

% (utf-8-ra kódolhatjuk az iso-8859-est
% az enconv paranccsal [enca csomag])
% utf-8 kódolás esetén ez a rész kell:

\usepackage{ucs}
\usepackage[T1]{fontenc}
\usepackage[utf8x]{inputenc}

%%%%%%%%%%%%%%%%%%%%%%%%%%%%%%%%%%%%%%%%%%%%%%%%%%
%%%% A MAGYAR NYELVVEL kapcsolatos csomagok
\usepackage[magyar]{babel}
\usepackage{indentfirst}  % Az első bekezdést is behúzza.

\frenchspacing   % A mondatvégek után azonos szóköz van, mint máshol
 % (Az angolban nagyobb a szokás. Ez az alapbeállítás.)

% A pdf kisebb és olvashatóbb lesz, ha times betűket használok sima
% (nem pdf-) latex esetén.
% \usepackage{times}

%%%%%%%%%%%%%%%%%%%%%%%%%%%%%%%%%%%
%%%% Az ÁBRÁKHOZ hasznos csomagok

%\usepackage{graphicx} % ábrák beillesztésének bővebb paraméterezése
% \usepackage{psfrag} % PostScript ábrák feliratainak cserélése TeX-esre

%%%%%%%%%%%%%%%%%%%%%%%%%%%%%%%%%%%%%%%%%%%%%%
%%%% MATEKKAL KAPCSOLATOS CSOMAGOK, DEFINÍCIÓK

% Az Amerikai Matematikai Társulat (AMS)
% hasznos csomagja  (pl. \dfrac)
%% \usepackage{amsmath}

% Magyar függvénynevek
%% \DeclareMathOperator{\tg}{tg}
%% \DeclareMathOperator{\sh}{sh}
%% \DeclareMathOperator{\ch}{ch}
%% \DeclareMathOperator{\cth}{cth}
%% \DeclareMathOperator{\ctg}{ctg}
%% \DeclareMathOperator{\arctg}{arctg}
%% \DeclareMathOperator{\arcctg}{arcctg}
%% \DeclareMathOperator{\arsh}{arsh}
%% \DeclareMathOperator{\arch}{arch}
%% \DeclareMathOperator{\arth}{arth}
%% \DeclareMathOperator{\arcth}{arcth}

% Ezzel a duplaszárú betűk elérhetőek.
% Pl. valós számok halmazjele:  \mathbb{R}
% (Nem tökéletes, mert mindkét oldalon dupla.)
%\usepackage{amssymb}

%%%%%%%%%%%%%%%%%%%%%%%%%%%%%%%%%%%%%%%%%%%%%%%%%%%%
%%%% KIMONDOTTAN EHHEZ A FÁJLHOZ DEFINIÁLT PARANCSOK
%\usepackage{html} %html-címekhez hasznos

%pygments-hez
\usepackage{fancyvrb}
\usepackage{color}
\newcommand\at{@}
\newcommand\lb{[}
\newcommand\rb{]}
\newcommand\PYbg[1]{\textcolor[rgb]{0.00,0.50,0.00}{\textbf{#1}}}
\newcommand\PYbf[1]{\textcolor[rgb]{0.73,0.40,0.53}{\textbf{#1}}}
\newcommand\PYbe[1]{\textcolor[rgb]{0.40,0.40,0.40}{#1}}
\newcommand\PYbd[1]{\textcolor[rgb]{0.73,0.13,0.13}{#1}}
\newcommand\PYbc[1]{\textcolor[rgb]{0.00,0.50,0.00}{\textbf{#1}}}
\newcommand\PYbb[1]{\textcolor[rgb]{0.40,0.40,0.40}{#1}}
\newcommand\PYba[1]{\textcolor[rgb]{0.00,0.00,0.50}{\textbf{#1}}}
\newcommand\PYaJ[1]{\textcolor[rgb]{0.73,0.13,0.13}{#1}}
\newcommand\PYaK[1]{\textcolor[rgb]{0.00,0.00,1.00}{#1}}
\newcommand\PYaH[1]{\fcolorbox[rgb]{1.00,0.00,0.00}{1,1,1}{#1}}
\newcommand\PYaI[1]{\textcolor[rgb]{0.69,0.00,0.25}{#1}}
\newcommand\PYaN[1]{\textcolor[rgb]{0.00,0.00,1.00}{\textbf{#1}}}
\newcommand\PYaO[1]{\textcolor[rgb]{0.00,0.00,0.50}{\textbf{#1}}}
\newcommand\PYaL[1]{\textcolor[rgb]{0.73,0.73,0.73}{#1}}
\newcommand\PYaM[1]{\textcolor[rgb]{0.74,0.48,0.00}{#1}}
\newcommand\PYaB[1]{\textcolor[rgb]{0.00,0.25,0.82}{#1}}
\newcommand\PYaC[1]{\textcolor[rgb]{0.67,0.13,1.00}{#1}}
\newcommand\PYaA[1]{\textcolor[rgb]{0.00,0.50,0.00}{#1}}
\newcommand\PYaF[1]{\textcolor[rgb]{1.00,0.00,0.00}{#1}}
\newcommand\PYaG[1]{\textcolor[rgb]{0.10,0.09,0.49}{#1}}
\newcommand\PYaD[1]{\textcolor[rgb]{0.25,0.50,0.50}{\textit{#1}}}
\newcommand\PYaE[1]{\textcolor[rgb]{0.63,0.00,0.00}{#1}}
\newcommand\PYaZ[1]{\textcolor[rgb]{0.00,0.50,0.00}{\textbf{#1}}}
\newcommand\PYaX[1]{\textcolor[rgb]{0.00,0.50,0.00}{#1}}
\newcommand\PYaY[1]{\textcolor[rgb]{0.73,0.13,0.13}{#1}}
\newcommand\PYaR[1]{\textcolor[rgb]{0.10,0.09,0.49}{#1}}
\newcommand\PYaS[1]{\textcolor[rgb]{0.25,0.50,0.50}{\textit{#1}}}
\newcommand\PYaP[1]{\textcolor[rgb]{0.49,0.56,0.16}{#1}}
\newcommand\PYaQ[1]{\textcolor[rgb]{0.40,0.40,0.40}{#1}}
\newcommand\PYaV[1]{\textcolor[rgb]{0.00,0.00,1.00}{\textbf{#1}}}
\newcommand\PYaW[1]{\textcolor[rgb]{0.73,0.13,0.13}{#1}}
\newcommand\PYaT[1]{\textcolor[rgb]{0.50,0.00,0.50}{\textbf{#1}}}
\newcommand\PYaU[1]{\textcolor[rgb]{0.82,0.25,0.23}{\textbf{#1}}}
\newcommand\PYaj[1]{\textcolor[rgb]{0.00,0.50,0.00}{#1}}
\newcommand\PYak[1]{\textcolor[rgb]{0.73,0.40,0.53}{#1}}
\newcommand\PYah[1]{\textcolor[rgb]{0.63,0.63,0.00}{#1}}
\newcommand\PYai[1]{\textcolor[rgb]{0.10,0.09,0.49}{#1}}
\newcommand\PYan[1]{\textcolor[rgb]{0.67,0.13,1.00}{\textbf{#1}}}
\newcommand\PYao[1]{\textcolor[rgb]{0.73,0.40,0.13}{\textbf{#1}}}
\newcommand\PYal[1]{\textcolor[rgb]{0.25,0.50,0.50}{\textit{#1}}}
\newcommand\PYam[1]{\textbf{#1}}
\newcommand\PYab[1]{\textit{#1}}
\newcommand\PYac[1]{\textcolor[rgb]{0.73,0.13,0.13}{#1}}
\newcommand\PYaa[1]{\textcolor[rgb]{0.50,0.50,0.50}{#1}}
\newcommand\PYaf[1]{\textcolor[rgb]{0.25,0.50,0.50}{\textit{#1}}}
\newcommand\PYag[1]{\textcolor[rgb]{0.40,0.40,0.40}{#1}}
\newcommand\PYad[1]{\textcolor[rgb]{0.73,0.13,0.13}{#1}}
\newcommand\PYae[1]{\textcolor[rgb]{0.40,0.40,0.40}{#1}}
\newcommand\PYaz[1]{\textcolor[rgb]{0.00,0.63,0.00}{#1}}
\newcommand\PYax[1]{\textcolor[rgb]{0.60,0.60,0.60}{\textbf{#1}}}
\newcommand\PYay[1]{\textcolor[rgb]{0.00,0.50,0.00}{\textbf{#1}}}
\newcommand\PYar[1]{\textcolor[rgb]{0.10,0.09,0.49}{#1}}
\newcommand\PYas[1]{\textcolor[rgb]{0.73,0.13,0.13}{\textit{#1}}}
\newcommand\PYap[1]{\textcolor[rgb]{0.00,0.50,0.00}{#1}}
\newcommand\PYaq[1]{\textcolor[rgb]{0.53,0.00,0.00}{#1}}
\newcommand\PYav[1]{\textcolor[rgb]{0.00,0.50,0.00}{\textbf{#1}}}
\newcommand\PYaw[1]{\textcolor[rgb]{0.40,0.40,0.40}{#1}}
\newcommand\PYat[1]{\textcolor[rgb]{0.10,0.09,0.49}{#1}}
\newcommand\PYau[1]{\textcolor[rgb]{0.40,0.40,0.40}{#1}}


\usepackage{tabularx}
\usepackage{url}
\newcommand{\alapcim}{http://mail.roik.bmf.hu/linux}
\newcommand{\code}{\texttt}
\newcommand{\dir}{\code}
\newcommand{\Esc}{\framebox{\texttt{Esc}}}
\newcommand{\Ctrl}{\framebox{\texttt{Ctrl}}}
\newcommand{\Alt}{\framebox{\texttt{Alt}}}
\newcommand{\Enter}{\framebox{\texttt{Enter}}}

\usepackage{tikz}
\usetikzlibrary{positioning,automata}
\tikzset{ %inner sep = 0.5mm,
  %minimum size = 5mm,
  >=latex,
  every state/.style={draw=blue!50,thick,fill=blue!20,minimum size=1.8cm},
  {small point/.style} ={circle,draw=blue!60,fill=blue!30,inner sep=0, minimum size= 1mm},
  box/.style ={circle,draw=black!60,fill=black!20,inner
  sep=5pt,fill=blue!30,draw=blue!60,rectangle},
  {edge/.style} ={thick,draw},
  {dedge/.style} ={draw,->},
 }

\begin{document}

\title{Linux alkalmazása}
\author{Horváth Árpád}
\maketitle

\section{Előtörténet}

A UNIX-ot 1969-ban fejlesztették ki  AT\&T vállalatnál. 1972-ben
ugyanitt fejlesztették ki a C programozási nyelvet, és a korábban
assemblerben írt UNIX-ot újraírták C-ben. Kezdetben szabadon
terjeszthető volt forráskóddal együtt, majd üzletet látva benne
zárt kóddal pénzért árulták. Több változata lett System V (V=öt), BSD
(Berkley Software Distribution, a Berkley egyetemről), HP-UNIX, SUN
Solaris, Mac OS X. Ezek egy része utóbbi később szabad szoftverré vált:
*BSD (pl. FreeBSD), OpenSolaris.

A  Szabad Szoftver Alapítvány (FSF) által 1984-ben indított GNU projekt
célja egy teljes Unix-kompatibilis szoftverrendszer kifejlesztése volt.
Ebből származik a Linuxra épülő terjesztések fordítóinak,
szövegszerkesztőinek, parancsértelmezőinek (shell-jeinek) jelentős
része.  És az olyan UNIX-os parancsok nyílt forráskódú változata, mint
például az  ls, cp parancsok.
Az alapítvány alapítója és a mozgalom ,,élharcosa'' Richard Stallman (rms).

1987-ben egy egyetemi tanár, Andy Tanenbaum, kifejlesztett oktatási
célokra egy minimális operációs rendszert, a MINIX-et. Ennek teljes
forráskódja megtalálható Tanenbaum: \textit{Operációs rendszerek} című
könyvében.

Ezt használta fel Linus Torvalds, svéd anyanyelvű finn egyetemi hallgató
egy operációs rendszer kernelének a kifejlesztéséhez, melyet később
Linuxnak neveztek el és azóta is Linus fejleszti. A munkát 1991-ben
kezdte.

Precíz szóhasználatban a Linux a kernelt jelenti. Teljes operációs
rendszerré a Linuxot a GNU-s programok teszik, ezt az operációs
rendszert GNU/Linuxnak szokták hívni. A különböző Linux terjesztések
ezeket a programokat szoftvercsomagokba szervezik és csomagkezelő
programokat használnak a szoftvercsomagok telepítésének egyszerűbbé
tételéhez.

A Linuxok (talán kizárólag) az X Window rendszert használják a grafikus
felületek alapjaként, ez biztosítja az alapvető keretrendszert GUI
környezetek építéséhez: ablakok kirajzolása és mozgatása a képernyőn,
együttműködés az egérrel és billentyűzettel. 

Erre épülnek rá az ablakozó rendszerek (pl. IceWM, (Open/Flux)Box) és a
összetettebb grafikus felhasználói felületek (GNOME, KDE, Xfce). 
{\small A kettő közötti pontos különbséget megnézhetné valaki, az
előbbiekben nincs kölcsönhatás az ablakok között (drag-and-drop,
vágólap).}

A GNU/Linux rendszerek és a UNIX-ok között jelentős hasonlóság van, amit
több szabvány, például a POSIX nagymértékű betartása tesz lehetővé. Aki
egy UNIX-rendszert ismer, az nem fog elveszni a Linuxban és fordítva.

\section{A szabad szoftver -- nyílt forrású szoftver}

A Szabad Szoftver Alapítvány a szabad szoftvert nem egyszerűen a szabad
(jogtisztán ingyenes) felhasználást érti, tehát nem a shareware
programokat, hanem a következő jogok meglétét:
\begin{itemize}
\item 0. szabadság: A program tetszőleges célra \emph{szabadon
felhasználható}.
\item 1. szabadság: A program működése \emph{szabadon tanulmányozható,
és módosítható} a saját célunknak megfelelően.
\item 2. szabadság: Szabadon \emph{terjeszthető és továbbadható}.
\item 3. szabadság: Lehetőség van a szoftver \emph{továbbfejlesztésére} és a
fejlesztés \emph{közreadására}.
\end{itemize}

Az 1-3. szabadág feltételezi a \emph{forráskód elérhetőségét}.

A szabad szoftver nem jelent feltétlenül ingyenességet. Ha például
valaki hajlandó azért fizetni, hogy egy szabad szoftvert az ő igényeire
alakítson valaki, akkor nyugodtan kérhet pénzt a fejlesztő.
Mindenesetre nem lehet visszaélni annyira vele, mint egy zárt forráskódú
program esetén, hiszen kidobhatom a fejlesztőt, ha túl sokat kér, és
másikat vehetek fel.

A szoftverek nagy csoportja a \emph{nyílt forrású szoftverek}, amelyek
nem feltétlenül szabad szoftverek, mert elképzelhető, hogy a fenti
feltételeket nem teljesítik. Tehát a nyílt forrású szoftverek
részhalmaza a szabad szoftverek.

Bővebben itt: \url{http://www.fsf.hu/index.php/Whatis_fs}.

\section{Linux-terjesztések (distribution)}

Grafikus felületek: KDE, talán a leghasználhatóbb kezdőknek, sokan
jobban szeretik a GNOME-ot, mert nincs annyi felesleges csicsa, az
előbbi a Qt grafikus könyvtárra (C++), az utóbbi a Gtk grafikus
könyvtárra (C) épül. Vannak kisebb erőforrásúak is mint például az
IceWM, LXDE, az XFCE és az FVWM. Nem Windows-szerű, de állítólag nagyon
használható: *box (Fluxbox, Openbox\dots), Enlightenment.

Linux-terjesztések (distributions):
\begin{itemize}
\item Debian: szervernek szeretik (meg a FreeBSD-t), deb csomagok
\item Ubuntu: egyszerű telepítés és használat (sima: GNOME-os, Xubuntu:
	gyors Xfce-vel, kubuntu: KDE-vel, edubuntu: oktatási programok)
	deb csomagok
\item SuSE, RedHat $\rightarrow$ Fedora, Mandriva (rpm csomag)
\item Slackware, gentoo, arch: a csomagok a forráskódot vagy a forráskód
        letöltési helyét tartalmazzák a fordításhoz szükséges információkkal
\end{itemize}

Érdemes otthon telepíteni Linuxot. Az Ubuntut vagy kitartóbbaknak a
Debiant ajánlom.  http://ubuntu.hu http://debian.com Szívesen veszem, ha
valaki másikakkal kísérletezik (Gentoo, Arch, FreeBSD, PC-BSD
tapasztalatok érdekelnek).

Ha valakinek extra eszköze van, vagy soros egere, akkor Ubuntuból az
alternatives változatot, amúgy a live változatot töltse le. (Szervernek
természetesen a server változatot, az alapból grafikus felület nélküli.)
A live változat telepítés nélkül is elindul CD-ről, így meg lehet nézni,
szeretne-e ilyet az ember, milyen hardwert ismer fel. Lassú gépekre
lubuntut telepítsünk, az jóval gyorsabb. Mivel a live CD-re nem fér rá
minden nyelv csomagja ezért érdemes a magyar változatot letölteni, de
a magyar nyelvi csomagokat felrakhatjuk hálózatról is.

Ugyan nem Linuxok, de a BSD változatai (FreeBSD, NetBSD, OpenBSD,
PC-BSD,\ldots) is szabad szotverek. Amíg a Linuxot több összetevőből
rakják össze (GNU projekt, Linux-kernel), addig a BSD-k teljes alaprendszere
(kernel+operációs rendszer) együtt kerül fejlesztésre így jobban
összecsiszolt. A programoknál lehetőség van ugyan a binárisból történő
telepítésre, de a forráskódból történő telepítés is nagyon egyszerű, a
gentoo-hoz hasonló. A hardvergyártók egyelőre még kevésbé veszik
figyelembe a BSD-seket, ezért a harvertámogatottság kisebb mértékű, mint
a Linux esetén. Végül két érdekesség: 1) a Mac OS X operációs rendszer, az
Apple Macintoshának 2002-ben bevezetett operációs rendszere szintén
BSD alapú; 2) hogy a Debian 2011 februárjában kiadott 6.0-ás
verziójában választható olyan telepítő is, amely Linux kernel helyett a
FreeBSD kernelt használja.

\newpage
\section{Vim}

A Vim egy nagyon hatékony szövegszerkesztő (text editor), amely
karakteres felületen is működik, de van grafikus felületű változata is
(gvim). Ez utóbbi használható Windowson is, és van változata Mac OS X-re
is.

A Vim képes több puffert is kezelni, és azokat összehasonlítani
(vimdiff), azok között másolni. A Vim kezelni tudja az egeret is, de fő
ereje a billentyűzet használatában van, és Unix/Linux rendszereken
abban, hogy a pufferek tartalmát külső Unixos szűrőutasításokkal (lásd
\ref{sub:pipe} alfejezetben) könnyedén lehet szűrni.

A Vi egyik utódja, melytől azonban megkülönbözteti egy halom hasznos
lehetőség, például több szövegpuffer kezelése, szintaxiskiemelés,
többszörös visszavonás (undo).

Több fejlesztőkörnyezet és shell, mint például az Eclipse és a bash, nem
véletlenül tartalmaz olyan lehetőséget, hogy Vim-szerűen lehet velük
szerkeszteni a szöveget.

A legfontosabb, amit a Vimnél meg kell érteni, a módok. A módok egy
vázlatos állapotgáfja található alább.

\smallskip

\begin{tikzpicture}[node distance=1.5cm,/tikz/initial text=,
                initial distance=5ex]
        \node[state,initial] (n) at (0,0) {normal}; 
        \node[state] (c) [above=of n]
                {\parbox[c]{1.5cm}{\small command-line}}; 
        \node[state] (v) [right=2cm of n] {visual}; 
        \node[state] (i) [below=1.5cm of n] {insert}; 
        %\node (h) [below=.5cm of i] {:help vim-modes}; 
        \path[dedge] (n) edge[bend left=15] node[left,near end] {:,/,?} (c)
                edge[bend left=15] node[above] {v,V,Ctrl-V} (v)
                edge[bend right=15] node[left,near end] {iIaAoOc\ldots} (i);
        %\path[dedge] (-1.5cm,0) -- (n);
        \path[dedge] (i) edge[bend right=15] node[right,near start] {\Esc} (n);
        \path[dedge] (v) edge[bend left=15] node[below] {\Esc} (n);
        \path[dedge] (c) edge[bend left=15] node[right,near start]
        {\Esc,\Enter} (n);
\end{tikzpicture}

Általában a Vim normál módban indul. Command-line (parancssori) módba
kerülünk például, ha :-ot nyomunk, ekkor vim parancsokat írhatunk be,
amelyet egy \Enter\ után végrehajthatunk (vagy Escape-pel
megszakíthatunk). A parancs megjelenik az alsó sorban. A vim módokat
megkereshetjük például a help paranccsal \verb!:help vim-modes!

Szintén a Parancssori módhoz tartozik a keresés is. Előrefelé a keresést
normál módból /-el indíthatjuk. Ezután egy reguláris kifejezést írhatunk
be és ezt \Enter\ után megkeresi.

Ha be szeretnénk írni valamit, azt Insert módban tehetjük. Ebbe a módba
leggyakrabban az \verb!i! paranccsal jutunk. Ha beírtuk a szöveget
Escape-el juthatunk vissza Normál módba.

A visual módban tudjuk kijelölni a szöveg egy részét (karaktereket,
sorokat vagy téglalap alakú blokkot), hogy azt például töröljük,
átírjuk, módosítsuk, szűrjük egy szűrőutasítással, cseréljünk benne
szövegrészeket vagy köré írjunk valami környezetet.

\subsection{A Vim beállítása, sorok kezelése,
parancsismétlés, kilépés}

Vegyük végig a vimtutorból az első 3 leckét.
Lépjünk be a mail.roik.bmf.hu szerverre, futtassuk a vimtutor parancsot.

Tanuljuk meg a blokkok kezelését a vimtutorból.
Lépjünk be a mail.roik.bmf.hu szerverre, futtassuk a vimtutor parancsot,
keressünk rá a téglalap szóra ( \code{/téglalap} ) és hajtsuk végre az
ott található feladatokat (9. lecke).

Hozzunk létre egy másolatot a \code{.bashrc} fájlról \code{tmp.sh} néven és
gyakoroljunk rajta a vim-mel.
\begin{Verbatim}
cp .bashrc tmp.sh
vim tmp.sh
\end{Verbatim}

Állítsuk be, hogy színezzen (syntax highlighting): \code{:syntax on}

Állítsuk be, hogy ne kelljen minden alkalommal megadni! Ha nincs
\verb+~/.vimrc+ hozzuk létre, és írjuk bele, hogy \code{syntax on},
esetleg a következő fájlt menthetjük \verb+~/.vimrc+ néven:
\url{http://www.arek.uni-obuda.hu/linux/vim/vimrc}

Még jobb követni a \url{http://github.com/horvatha/vimrc} oldal alján
levő leírást a tároló klónozásához, a .vimrc létrehozásához és a hasznos
kiegészítők telepítéséhez.

Sorok törlése (delete): \code{dd}, parancs ismétlése \verb+.+ (pont),
törölt sor (szó) beillesztése aktuális sor (karakter) után \code{p}
(paste), elé \code{P}.
Hogyan lehet leggyorsabban két sort felcserélni?

Szöveg beírása: \code{i} (insert) után lehet, utána \Esc\ a normál módba
visszatéréshez. Ekkor szerkeszthetjük, ahogy más szerkesztőkben szoktuk.
(Az \code{i} vált át normál módból beszúrás (insert) módba, amit
megfelelő beállítások esetén alul láthatunk (\verb+--INSERT--+), az
\Esc\ tér vissza.)

Parancs n-szeri végrehajtásához előtte beírni a számot. Próbáljuk ki:

\code{3dd5$\downarrow$2p}

\code{22i+-} \Esc

A visszavonásokat az \code{u} -val tehetjük meg. Nyomjuk le többször az
\code{u}-t. Mi történik?

Mentsük a fájlt \code{:w}, más néven is: \code{:w fajlnev}! Próbáljuk
meg mégegyszer ugyanolyan néven! De csak azért is mentsük \code{:w! fajlnev}!

\textbf{Fontos:} Kilépési lehetőségek: mentéssel \code{:wq}, mentés nélkül
\code{:q!}. Ha nem vagyunk biztosak, milyen módban vagyunk, nyomjunk
előtte két \Esc-et

A \code{d} helyett \code{y}-t (yank=ránt) használva nem törlünk, hanem csak
másolunk. Próbáljuk ki több sor másolását több példányban!

Ez csak a kezdet. Apránként ismerkedünk majd meg a Vim finomságaival.
Az eddigiek közül a szintaxiskiemelés nem működik sima vi-jal és az
\code{i} parancs után kissé nehezebben lehet szerkeszteni a fájlt.

Egyedül tanuláshoz:

Írjuk be a \code{vimtutor} parancsot :-)

magyarul: http://people.inf.elte.hu/birki/ab/gyak/02.txt.html

angolul: \code{vimtutor} parancs végigvezet sok hasznos tudnivalón.

angolul: A vim helpjében sok hasznos dolgot találunk:\\ \verb!:help user-manual!

angolul: fontosabb parancsok http://worldtimzone.com/res/vi.html

A Vim letölthető Windowsra is (gVim): vim.org

\newpage
\section{Parancsuralom a shell felett}

Töltsük le a Linuxos anyagot a bazaar verziókezelő rendszerrel:
\begin{Verbatim}[commandchars=@\[\]]
@PYaD[#] apt-get install bzr    # Ha még nincs bazaar
@PYaD[$] cd ahova/le/akarom/tölteni
@PYaD[$] bzr branch http://mail.roik.bmf.hu/linux
\end{Verbatim}

A bazaar részletesebb használatát lásd a Bazaar szakaszban
(\ref{seq:bazaar}. szakasz).

Fordítsuk le és nyissuk meg a gyakorlatok leírását
(F11 teljes képernyő):
\begin{Verbatim}
cd segedlet
make
evince linalk.pdf&
\end{Verbatim}
Az \& háttérbe küldi a folyamatot, és visszakapjuk a promptot, további
utasításokat írhatunk be az előző folyamat megszakítása nélkül.

\subsection{BASH: fájl- és könyvtárműveletek, jogok, beállításai}

A \code{bash} a Linux/UNIX egyik héja (shellje, parencsértelmezője),
amely parancsok végrehajtását teszi lehetővé. Van még sh, csh, zsh,
ksh, dash\dots

Készítsünk részletes listát a fájlokról, könyvtárakról, a rejtetteket is
beleértve:
\code{ls -la}

A rejtett fájlok/könyvtárak ponttal kezdődnek, alapból nem jelennek meg
listázáskor, általában beállításokat tartalmaznak.

A fájllista elején szerepelnek a jogosultságok pl. \code{drwxr-x---}
első betűje könyvtárat jelöl, a következő három betű a felhasználó
jogait (olvasás/listázás, írás és végrehajtás/belépés) a következő
kétszer három a csoport ill. a többi felhasználó jogait.

A következő oszlop még nem érdekes, a hardlinkek számát mutatja.
Utána áll a fájl/könyvtár tulajdonosa és csoportja, amire a (jogok
vonatkoznak), majd a méret, módosítási dátum és a fájlnév.

Keressük meg az ls leírásában a \code{-l -A -a -t} és \code{-r} kapcsolókat:
\begin{Verbatim}
man ls
/-l<Enter>
\end{Verbatim}
A man-ból q-val léphetünk ki.

Mit jelent az \code{ls -trl}?

Váltsunk könyvtárat \code{cd /}, listázzuk, térjünk vissza \code{cd}
(call directory)

Ha a \code{manpages-hu} csomag telepítve van, akkor a leggyakoribb
utasításokról magyarul kapunk leírást.

Nézzük meg a \code{mv} parancs leírását. Mit csinál, milyen formában
használhatjuk? Mit jelent a forrás... (a három pont) az áttekintésben?

Hozzunk létre és tegyünk futtathatóvá a bin könyvtárban a
\code{cx} parancsot, melynek tartalma:
\begin{verbatim}
chmod 755 $*
\end{verbatim}

Állítsuk be, hogy a \code{bash} a gép minden indulásakor megtalálja a
home-könyvtárunkban található \dir{bin} könyvtárat!

Állítsuk be a hasznos \code{ll, la} parancsokat, hogy mindig
rendelkezésre álljanak (alias)!

Indítsunk új terminált (vagy újból jelentkezzünk be) és próbáljuk ki az
\code{ll} és \code{cx} parancsainkat! Állapítsuk meg a típusukat!
\code{type ll cx}

Fontos parancsok: cp (copy), ls (list), mv (move), rmdir, mkdir, pwd,
chmod, chown, alias, type

Próbáljuk kitalálni a még ismeretlenek jelentését! Hozzunk létre egy
temp nevű könyvtárat, majd töröljük! Hozzuk létre megint (felfelé
gomb=előző utasítás), másoljuk bele a tmp fájlunkat! Most töröljük!
(\code{rm -r}, rekurzív)

Otthoni tanuláshoz magyarul:\\
\url{http://www.cab.u-szeged.hu/local/doc/UNIX/orlando/bev.html} Orlando
Unix-iskola

Bevezetés a bashbe angolul:\\
\url{http://www.tldp.net/LDP/abs/html/index.html} Advanced bash-scripting guide

\subsection{Csővonal (pipe) és átirányítás}
\label{sub:pipe}

% subsection subsection name (end)
A UNIX-os utasítások egy része szűrő jellegű, ami háromféleképp veheti a
bemenetét.
\begin{enumerate}
\item egy csővonalból
\item a billentyűzetről
\item egy fájlból
\end{enumerate}

Igazából az első kettő a UNIX/Linux szempontjából ugyanaz. Ő ha nincs
megadva fájl, az úgynevezett standard bemenetről (stdin) várja az adatot, ami
vagy az úgynevezett csővonalból jön, vagy a billentyűzetről. A hasznos
kimenetet minden szűrő a standard kimenetre (stdout) küldi, ez
csővonallal továbbküldhető egy másik utasításnak, vagy megjelenik a
képernyőn, esetleg beleirányíthatjuk egy fájlba. Ha egy utasítás
végrehajtása során hiba történik, az nem megy a csővonalba, hanem az
úgynevezett standard hibakimenetre (stderr).

Hogy értsük is, pár példát mutatunk: az egyik szűrőutasítás a
\code{grep}. Ez egy szövegrészletet keres meg a bemenetében, pontosabban
egy reguláris kifejezésre keres rá.

\paragraph{1a. példa}

\verb+ls /usr/bin | grep im$+

Itt az ls parancs kilistázza a felhasználók által végrehajtható
utasításokat és kiírná a képernyőre (stdout-ra) a kimenetet, de a |
(függőleges vonal) azt mondja, hogy a kimenetet adja át a grep
utasításnak, ez lesz az ő stdin-je. A grep pedig kiszűri az im végű
fájlneveket. A \verb+$+ a sor végét jelenti a reguláris kifejezésekben.


\begin{tikzpicture}[scale=2]
 \node[box] (ls) at (0,0) {ls};
 \node[small point] (P) at (1,0) {};
 \node[box] (grep) at (2,0) {grep};
 \node[box] (scr) at (4,0) {képernyő};
 \path[dedge] (ls) edge node[above] {stdout}  (P);
 \path[dedge] (P) edge node[above] {stdin}  (grep);
 \path[dedge] (grep) edge node[above] {stdout}  (scr);
\end{tikzpicture}

\paragraph{1b. példa}

\verb+ls /usr/bin | grep im$ >ezegyfajl+

Az előzőhöz hasonló, csak a kimenetet egy fájlba küldjük, nem a
képernyőre.

\paragraph{2. példa}

\verb+grep im$+

Ilyenkor a billentyűzetről vár bemenetet, és minden soremeléskor, ha a
sor illeszkedik a mintára, kiírja a sort.

\begin{tikzpicture}[scale=2]
 \node[box] (kb) at (0,0) {billentyűzet};
 \node[box] (grep) at (2,0) {grep};
 \node[box] (scr) at (4,0) {képernyő};
 \path[dedge] (kb) edge node[above] {stdin}  (grep);
 \path[dedge] (grep) edge node[above] {stdout}  (scr);
\end{tikzpicture}

A futtatása így néz ki:
\begin{verbatim}
$ grep im$
sünim
sünim
image
Palim
Palim
FELEIM
\end{verbatim}

A végén Ctrl-D vel jelezzük, hogy végeztünk (ezzel fájlvége jelet, EOF-et
küldünk).


\paragraph{3. példa}

\verb+grep ^alias ~/.bashrc+

Ha egy fájlnév adott a szűrőutasításban, akkor a fájlból veszi az
adatokat. Itt a alias-szal kezdődő sorokat. A \verb+^+ (kalap jel) a sor
elejére illeszkedik.

\begin{tikzpicture}[scale=2]
 \node[box] (file) at (0,0) {fájl};
 \node[box] (grep) at (2,0) {grep};
 \node[box] (scr) at (4,0) {képernyő};
 \path[dedge] (file) edge node[above] {stdin}  (grep);
 \path[dedge] (grep) edge node[above] {stdout}  (scr);
\end{tikzpicture}

A futtatása így nézhet ki:
\begin{verbatim}
$ grep ^alias ~/.bashrc 
alias lt='ls -ltr'
alias ll='ls -l'
alias la='ls -A'
alias cx="chmod a+x"
alias cr="chmod a+r"
alias ipythonlab='ipython -pylab'
\end{verbatim}

\paragraph{4. példa}

\verb+grep ^alias nincsilyenfajl | sort+

Ha a grep-ben valamilyen hiba van, például nem létezik a megadott
nevű fájl, akkor a kimenete (a hiba jelzése) nem megy tovább a
csővonalon, hanem a stderr-re megy, és mindenképpen kiíródik. (Na jó,
van arra is mód hogy átirányítsam a hibakimenetet is egy fájlba.) Így a
sort szűrő, ami rendezni szokott nem nyeli el a hibautasítást (ekkor azt
hinném, minden rendben volt), és nem hibautasítást fog rendezni.

\begin{tikzpicture}[scale=2]
 \node[box] (file) at (0,0) {nem létező fájl};
 \node[box] (grep) at (2,0) {grep};
 \node[box,fill=red!20] (scr) at (2,-1) {képernyő};
 \node[box] (sort) at (4,0) {sort};
 \path[dedge, dashed] (file) edge node[above] {stdin}  (grep);
 \path[dedge, dashed] (grep) edge node[above] {stdout}  (sort);
 \path[dedge] (grep) edge node[right] {stderr}  (scr);
\end{tikzpicture}

\paragraph{További szűrőutasítások}

Milyen héjaink (shelljeink) vannak?
\begin{Verbatim}
ls /bin |grep sh
\end{Verbatim}

A \code{cat fn} utasítással kiírhatunk egy \code{fn} fájlt, vagy többet. Keressük meg ezzel
a diakok sorait a \code{/etc/passwd} fájlban! Használjuk a grep-et, hogy csak a
diak tartalmú sorokat írja ki.

Számoljuk meg, hány ilyen sor van! wc (word count, sor-szó-betűszám) -l
kapcsolóval csak sorok száma.
\begin{Verbatim}
cat /etc/passwd |grep diak|wc
\end{Verbatim}

Mivel a grep fájlból is olvashat ezt így is végrehajthatnánk:
\begin{Verbatim}
grep diak /etc/passwd |wc
\end{Verbatim}
Ha viszont több fájlból kell szűrnünk, akkor a cat nélkül nem megy:
\begin{Verbatim}
cat /etc/passwd /etc/group |grep diak
\end{Verbatim}


Vajon hogyan számolhatjuk meg a sorok számát a passwd fájlban? És
\code{/home} fájljainak számát (azaz általában a felhasználók számát)?

Szűrők és fontosabb kapcsolóik (minuszt (-) tegyünk a kapcsolók elé):\\
\begin{tabular}{|l|l|l|}
\hline
neve& leírás & kapcsoló\\
\hline
grep & szűri a sorokat egy minta szerint&v r\\
\hline
sort &rendezi a sorokat (alapból abc-rendbe)&n u c\\
\hline
wc & szószámláló (word count)&l c w\\
\hline
head & csak a fájl elejét írja ki (=fej)&n c\\
\hline
tail & csak a fájl végét írja ki (=farok)&n c\\
\hline
sed & pl. szócseréket végez a szövegben (stream editor)&\\
\hline
awk & összetett feladatok végzésére (a szerzők nevéből)&\\
\hline
tr & karaktercsere&\\
\hline
\end{tabular}

Az alábbi példákat a homekönyvtárban próbáljuk ki:
\begin{Verbatim}
grep alias .*
cat /etc/passwd | sort
ls /bin | wc -l
head .bashrc
tail -5 .bashrc
ls -l ~ | sed "s/diak/****/g"
cat .bashrc | tr eat EAT
cat /etc/fstab | awk '/dev/ {print NR, $1, $3}'
\end{Verbatim}
Keressük meg a man-ban a fenti utasítások táblázatban szereplő
kapcsolóit. (pl. -l keresése: /-l)

A sed csereutasítása a Vimben is hasznos. A head és tail nagy fájloknál is
gyorsan működik. Az awk egy önálló programnyelvvel rendelkező szűrő.
Jó leírása van a Kernighan--Pike könyvben, de a man-ban is vannak
hasznos példák.

A legvégső kimenet általában a képernyőre kerül.  Bárminek a kimenete
cső vagy a képernyő  helyett fájlba is helyezhető $>$ segítségével.
Pl:
\begin{Verbatim}
cat .bashrc .bash_profile > egyesitett
(df; uname -a) >gep_adatai
\end{Verbatim}

\subsection{ClusterSSH}
Ha egyszerre több gépen -- például egy cluster gépein -- kell
ugyanazokat a parancsokat végrehajtanom, akkor hasznos program a
ClusterSSH. Futtatásához, grafikus felület kell. A \texttt{cssh}
parancsnak paraméterként megadva a szerverek IP-címének listáját
mindegyik szerver terminálja külön ablakban jelenik meg, és egy kis
beviteli mező is megjelenik, amelyben egyszerre az összes szervernek
adhatunk parancsot.  Az egyes terminálokba belépve külön is
foglalkozhatunk egyetlen géppel is.

Telepítése Debian és Ubuntu rendszereken:
\begin{Verbatim}
# apt-get install clustterssh
\end{Verbatim}

\newpage
\section{\code{.deb} csomagok telepítése és kezelése}

A telepítési parancsokat root-ként adhatjuk ki, vagy megfelelő
beállítások esetén \code{sudo}-t írva eléjük (Ubuntuban ez az alap,
\code{man sudo}).

A Linuxokon a programok, könyvtárak csomagokba vannak telepítve. Egy
könyvtárat több program is használhat. A \code{dpkg} egyedi csomagok
telepítésére  (\code{-i}), telepített csomagok listázására (\code{-l}),
csomagok újrakonfigurálására (\code{dpkg-reconfigure}).
Gyakran hasznos a \code{dpkg -l|grep kulcsszó} alak csomagok keresésére.

Az apt a dpkg-re épül, függőségeket, csomagok letöltését kezeli.
Azt, hogy honnan tölthet le, a \code{/etc/apt/sources.list} szöveges
fájl tartalmazza. Az alábbi első sor frissíti a csomaglistát, a második
letöltögeti az \code{mc}, \code{vim-full} és \code{gimp} csomagokat
függőségekkel együtt és telepíti.
\begin{Verbatim}
apt-get update
apt-get install mc gimp vim-full
\end{Verbatim}
Bővebben: \url{http://321.hu/Debian/apt-guide.html}

Az alábbi sorokkal kereshetünk kulcsszónak megfelelő csomagokat, illetve
megnézhetjük a csomag leírását:
\begin{Verbatim}
apt-cache search postgresql  |grep python
apt-cache show vim
\end{Verbatim}

Az utóbbi kimenetének részlete:\\
{\small
Package: vim\\
Priority: optional\\
Provides: editor\\
Depends: libc6 ($>$= 2.7-1), libgpmg1 ($>$= 1.19.6-1), libncurses5 ($>$= 5.6+20071006-3), python2.5 ($>$= 2.5), vim-common (= 1:7.1-138+1ubuntu3), vim-runtime (= 1:7.1-138+1ubuntu3)\\
Filename: pool/main/v/vim/vim\_7.1-138+1ubuntu3\_i386.deb\\
Description: Vi IMproved - enhanced vi editor\\
Origin: Ubuntu
} %small

Ebből láthatjuk, hogy milyen más csomagok melyik verziójától függ a {\tt
vim} csomag. A csomagok prioritása lehet required, important, standard,
optional and extra; ebben az esetben optional.
A csomagok pl. az Ubuntu alatt három fő tárolóban vannak, ezek nevei:
main, universe és multiverse.
A Filename kezdetű sorból láthatjuk, hogy a {\tt vim} a main tárolóban
található és az Intel 368-as (i386) architektúrára fordított csomagunk van.

Igazából a fenti kimenetet a \code{apt-cache show vim}  sorral
csináltuk. Az apt-get és apt-cache parancsok helyett gyakran írhatunk
aptitude-ot is (az alábbi összefoglalóan az upgrade-eket kivéve ugyanúgy),
hasonló információkat tudhatunk meg, kicsit más kinézettel. Az aptitude
talán többmindent magyarít, ha a magyar nyelv van beállítva.

Az apt grafikus felülete a Synaptic csomagkezelő (Rendszer $>$
Adminisztráció $>$ Synaptic). 

Hasznos parancsok röviden összefoglalva:
\begin{verbatim}
dpkg -i <deb-fájl>        # Egy deb-csomagot telepít.
dpkg -l                   # (esetleg greppel szűrve) Telepített csomagok listája
dpkg-reconfigure <csomag> # (xserver-xorg, tzdata) Csomagok újból-beállításai

apt-get update             # A tárolóra vonatkozó információk frissítése.
apt-get  install <csomag>  # Csomag telepítése (függőségeket feloldja)


apt-get upgrade       # A tárolóban frissült csomagok telepítése
aptitude safe-upgrade # Az előző megfelelője.

vim /etc/apt/sources-list  # A tárolók helyének megadása.

apt-get dist-upgrade  # Frissítés új verzióra.
aptitude full-upgrade # Az előző megfelelője.

apt-cache search <keresett_szó>  # Keresés a csomagnévben
                                 # illetve a rövid és a hosszú leírásban
apt-cache show <csomag>          # Adott csomag fontosabb adatai
\end{verbatim}

\section{Telepítés}
\input{debian-telepites}

\newpage
\section{Munka több gép között}

Az alábbiakban a távoli gépekkel történő kapcsolatokat vizsgáljuk. Ha
távoli gépen akarunk ugyanúgy dolgozni, mint a sajátunk karakteres
felületén, akkor azt általában ssh-val tesszük, ami titkosított
kapcsolatot létesít. Fájlok másolására az ftp, az sftp vagy az scp
szolgál. Leírás ezekről (Windowsos lehetőségek is):\\
http://progkor.inf.elte.hu/pandora.htm

\subsection{ssh}
Jelentkezzünk be a távoli gépen.\\
\texttt{ssh -p x diak\textit{n}@mail.roik.bmf.hu}\\
ahol $n=1,2,\dots,15$, a port x értékét nem írom ide.

Mostmár az alábbi kódolási eltéres ritkább, a szerverek és asztali
(desktop) gépek Linuxai is általában utf-8-as kódolást használnak, de
hátha mégis hasznos lesz valakinek.

Ha utf-8-as kódolású gépről jelentkezünk be latin2-esre (iso-8859-2),
akkor az ékezetes betűk nem mennek át. Ekkor hasznos az alábbi alak.
\begin{Verbatim}
luit -encoding iso-8859-2 -- ssh -p x diak1@mail.roik.bmf.hu
\end{Verbatim}
Windows alól a putty.exe programmal léphetünk be. Ott is beállítható a
karakterkódolás a \texttt{Translation} részben.

A fejezet többi része nem tartozik szorosan az ssh-hoz.
Váltsunk a gyökérkönyvtárra \code{cd /}, listázzuk, térjünk vissza \code{cd} 
(call directory) A diak15 felhasználó (home)könyvtárának rövidítése:
\code{\~{}diak15}, a sajátunké \code{\~{}}, a pillanatnyié \code{.}, a
pillanatnyi szülőkönyvtáráé \code{..} Váltsunk diak15 könyvtárába, majd
vissza!

Mennyi szabad hely van a lemezen: \code{df} (disc free). Mennyit használ
a pillanatnyi könyvtárunk: \code{du -h} (disk usage).

Nézzük meg ki mit csinál a gépen: \code{w} (who). Ha elfelejtem hol vagyok:
\code{pwd} (print working directory, egyébként azt többnyire látjuk a
prompt (\$) előtt). Milyen felhasználói könyvtárak
vannak: \code{ls /home} vagy (ha a sajátomban vagyok) \code{ls ..}

Indítsuk el a Vim oktatóját: \code{vimtutor hu}  Ennek a
végigtanulmányozása házi feladat. Kilépés q!
Ha valaki otthon akarja kipróbálni, és nem akar bejelentkezni a mail-re,
a \alapcim{}/vim oldalról letöltheti a tutor.hu
fájlt, és \code{vim tutor.hu} utasítással megnyithatja. (vim.org-ról
letölthető a windowsos vim telepítője, azzal is megfelel.)

\subsection{(s)ftp, scp}
A sima ftp-t használhatjuk, ha nem titkosított fájlokat kell
letöltenünk, pl. valamelyik Linux-disztribúció  iso-ját. Ellenkező
esetben sftp-t használhatunk ($\in$ lftp), nyilván a titkosítás és
visszafejtés időt és erőforrást igényel.

Az ftp leggyakoribb parancsaival ismerkedjünk meg. Jelentkezzünk be:
\begin{Verbatim}
ftp mail.roik.bmf.hu
\end{Verbatim}
Ez majd rákérdez a felhasználóra és a jelszóra. cd-vel válthatunk
távoli könyvtárat ls-sel listázhatunk, lcd-vel helyi könyvtárat
válthatunk. (A \code{!cd} parancs jogosan nem vált könyvtárat, hiszen
minden parancs egy saját héjat (shellet) nyit meg, és abban vált
könyvtárat, majd visszalép az eredeti héjba, ahol ez nem érvényes.) A !
elejű sorok helyi parancsként értelmeződnek. help segítséget ad.

Az (m)get és (m)put paranccsal lehet fájlokat lehívni és feltölteni. Az
m esetén több (multiple) fájlt, pl.  \code{mget *.jpg} Nagyon sok fájl
esetén érdemes a \code{prompt} utasítást kiadni, akkor nem kérdez rá
minden egyes fájlra letöltés előtt.

Töltsük le a mail szerverről a diak1 felhasználó kepek/kepek
könyvtárában levő képeket!

sftp szintaktikája:
\begin{Verbatim}
sftp diak1@mail.roik.bmf.hu
\end{Verbatim}

Amúgy a Firefox-szal is ftp-zhetünk. Beírhatjuk URL-nek:
\begin{Verbatim}
ftp://mail.roik.bmf.hu
\end{Verbatim}

Midnight Commanderben (mc) a menüből kiválasztható a Jobb/FTP-kapcsolat
menüpont. A Total Commander is képes ftp-re.

Az scp utasítást akkor használhatjuk, ha pontosan tudjuk a fájl nevét és
helyét.
Használata a cp utasításéhoz hasonló.  Ilyenkor csak a jelszóra kérdez
rá, nem ad promptot. Ez szkriptben hasznos. Pl:
\begin{Verbatim}
scp diak1@mail.roik.bmf.hu:.bashrc temp
\end{Verbatim}

\section{Elosztott verziókezelő rendszerek}

A központosított verziókezelő rendszerek (SVN, CVS) esetén minden
fejlesztésben résztvevő személy ugyanarra a szerverre küldi fel a
változásait, és a gépén csak a fájlokról van másolata.

Az elosztott verziókezelő rendszerek (git, bazaar, mercury) esetén a
fejlesztőknek a saját gépükön is megtalálhatóak azok az információk,
amelyek a tárolón (repository) vannak, így akár internetkapcsolat nélkül
is tud feltölteni (commit) a saját gépén lévő tárolójába változásokat új
változatot (revision) hozva létre, amit aztán -- megfelelő jogosultságok
esetén -- felküldheti (push) egy interneten elérhető szerverre.
Leszedheti (pull) és összefésülheti (merge) mások változataival.

Mindegyik fejlesztésben alapvető egység az ág (branch), melyben
fejlesztési modelltől függően a különböző verziókat tárolhatjuk, vagy
esetleg egy új megvalósítandó  is új ágat nyithatunk:
http://nvie.com/posts/a-successful-git-branching-model/ .

A bazaar alapegysége az ág, a git-é az akár több ágat is
tartalmazó tároló. A bazaar-ban egy új ág létrehozásához új könyvtárat
kell létrehozni, míg a git-ben az ágak váltásakor ugyanabban a
könyvtárban változik meg a fájlok tartalma.

A bazaar esetén tehát ágat (branch) másolok le, git esetén a tároló
klónját (clone) hozom létre.


\subsection{git}
A git elosztott verziókezelő rendszert Linus Torvalds fejlesztette ki a
Linux kernel fejlesztéséhez. Jól illeszkedik a https://github.com
oldalhoz.

Egy tárolóban több ág található, amelyek között gyorsan
válthatunk.

Letölthető illetve frissíthető vele például a linuxos tananyag
legfrisebb változata:

\begin{Verbatim}
# aptitude install git gitk
$ cd ahova/le/akarom/tölteni
$ git clone https://github.com/horvatha/linux.git
$ cd linux
\end{Verbatim}

Ha közben változik a githubon a tároló, ugyanebben a könyvtárban a
klónunkat frissíthetünk így:

\begin{Verbatim}
$ git pull
\end{Verbatim}

A \verb"#" a root-jogosultságot jelöli, helyette sudo is lehet.

Grafikus felületen áttakinthetőbb képet kapunk a változásokról így:
\begin{Verbatim}
$ bzr gitk
\end{Verbatim}

\subsection{Bazaar}
\label{seq:bazaar}
A bazaar elosztott verziókövető rendszert, az a Canonical feljeszti, aki
az Ubuntut is fejleszti. Az Ubuntu http://launchpad.net oldalával jól használható.

Letölthető vele például a linuxos tananyag legfrisebb változata:

\begin{Verbatim}
# aptitude install bzr
$ cd ahova/le/akarom/tölteni
$ bzr branch http://arek.uni-obuda.hu/repo/cxnet
\end{Verbatim}

Frissítéskor csak ennyi kell

\begin{Verbatim}
$ bzr pull
\end{Verbatim}

Ha változtattunk, és a változtatásainkat is meg szeretnénk őrizni, akkor
a következőket tegyük.

\begin{Verbatim}
$ bzr commit  # Ha változtattunk valamit
$ bzr merge http://arek.uni-obuda.hu/repo/cxnet
\end{Verbatim}

Grafikus felületen áttakinthetőbb képet kapunk a változásokról így:
\begin{Verbatim}
$ bzr visual
\end{Verbatim}

Feltölthetünk saját verziókövetett rendszerünket egy tetszőleges helyre,
amit pl. sftp-vel elérünk:
\begin{Verbatim}
bzr push sftp://diak1@mail.roik.bmf.hu:4522/home/diak1/public_html/linux
\end{Verbatim}
A 4522 a port száma. A szerver azóta nem létezik, de a példának jó. A
linux könyvtár, ha nem létezik, létrejön.

\newpage
\section{PostgreSQL}

Az alábbi leírás szövegfájlban: linux/postgresql/README.txt

\input{psqljegyzet}

% http://groups.google.com/group/linux.debian.bugs.dist/browse_thread/thread/5ddfcc41e022ff76/17817ee8a08671a1?lnk=raot
%xxx Az alábbival kéne megoldani, de ékezetes karaktereket nem veszi be
% jól  utf-8-ból és latin2-ből sem.
%\lstinputlisting{../postgresql/jegyzet}

\subsection{Elérése Pythonból}

Az alábbi programfálban nyomon követhetjük, hogyan érhetjük el Python
alól az adatbázisunkat:
\begin{Verbatim}[commandchars=@\[\]]
@PYaD[#! /usr/bin/python]
@PYaD[# coding: utf-8]
@PYay[from] @PYaV[__future__] @PYay[import] division

@PYas["""Egyszerű program az adatbázis elérésére Pythonból."""]

@PYaD[# python-pygresql csomag kell hozzá]
@PYaD[# (apt-get install python-pygresql)]

@PYaD[# http://www.pygresql.org/pgdb.html]

@PYay[import] @PYaV[pgdb]
@PYaD[# Ha más adatbázis van, akkor csak ezt a sort kell változtatni. Pl:]
@PYaD[# import oracledb]

connection @PYbe[=] pgdb@PYbe[.]connect(database@PYbe[=]@PYad[']@PYad[diak]@PYad['], user@PYbe[=]@PYad[']@PYad[diak]@PYad['],
		    password@PYbe[=]@PYad[']@PYad[diak]@PYad['], dsn@PYbe[=]@PYad[']@PYad[mail.roik.bmf.hu]@PYad['])

cursor @PYbe[=] connection@PYbe[.]cursor()
cursor@PYbe[.]execute(@PYad["]@PYad[SELECT * FROM betétesek;]@PYad["])
connection@PYbe[.]commit() @PYaD[# Csak SELECT parancs esetén nem fontos]

lekert_adatok @PYbe[=] cursor@PYbe[.]fetchall()

@PYay[for] nev, cim, egyenleg @PYan[in] lekert_adatok:
    @PYay[print] nev, cim, egyenleg
    @PYaD[#print nev.left(20), cim.left(20), egyenleg.left(5)]
    @PYaD[#print "{0:20} | {1:20} | {2:5}".format(nev, cim, egyenleg))]
\end{Verbatim}


Az utolsó megjegyzésbe helyzett sorokkal szebb formátumba tudjuk írni.
Próbáljuk megérteni az adott sorokat!

\section{\LaTeX}

Töltsök le githubról a horvatha/latex következő tárolót:

\begin{Verbatim}
git clone https://github.com/horvatha/latex.git
\end{Verbatim}

A \verb!latex/peldak/keret.tex! fájl jó kiindulófájlnak.

A \verb!latex/latex-roviden! könyvtárban a \verb!make! paranccsal
fordítsuk le a rövid összefoglalót. A benne leírtakat kell tudni
használni.

Ennek a fájlnak a forrása a linux tárolóban van, ha magunk akarjuk
lefordítani, töltsük le a tárolót:

\begin{Verbatim}
git clone https://github.com/horvatha/linux.git
\end{Verbatim}

A tikzpicture ábrákhoz szükség van a pgf csomag telepítésére:

\begin{Verbatim}
apt-get install pgf
\end{Verbatim}

Nyissuk meg ennek a fájlnak a forrását és fordítsuk le:
\begin{Verbatim}
cd linux/segedlet
vim linalk.tex
:!pdflatex linalk.tex
:!evince linalk.pdf &
\end{Verbatim}
A :! a vim-ben lefuttaja az utána lévő parancsot, mintha shellbe írnánk.

A pdflatex-et Vimben egyszerűbben is futtahatjuk, mivel a \verb!%! az
aktuális fájlt jelöli, sőt -- mivel a fenti fájl a Makefile-ban is benne
van -- make-kel is fordíthatom, de ilyenkor a célfájlt kell megadni:
\begin{Verbatim}
:!pdflatex %
:make linalk.pdf
\end{Verbatim}
Mint látjuk a make-hez nem kell felkiáltójel Vimben.

Újraforsításkor a régebbi Evincében a \verb!<Ctrl>+R! frissít.
Újabbakban automatikusan frissül.

Kis kedvcsinálónak nézzük meg a latex tárolóból a
\code{latex/latex-izelito/sziv.pdf} fájlt.
Ezek inkább érdekességek, hogy mit lehet csinálni a \LaTeX-hel.
A forrás is ott van \code{sziv.tex} néven. Szintén lefordíthatjuk

\verb"pdflatex sziv.tex"

utasítással.

Menjünk át a latex tároló latex-roviden könyvtárába, és fordítsuk le a
latex-roviden.tex fájlt pdflatex-hel!

Másoljuk át a latex-roviden.tex-et más névre, és a benne leírtak
és a forrás segítségével:
\begin{itemize}
\item Írjunk még pár bekezdést!
\item Hozzunk létre új fejezetcímeket!
\item Helyezzük át a tartalomjegyzéket máshová!
\item Töröljük a felesleges részeket!
\item Állítsuk be magunkat szerzőnek és írjuk át a címet!
\item Szúrjuk be a Pitagorasz-tétel képletét \verb+$+-ral és
 \code{equation} környezettel, valamint a trigonometrikus
 Pitagorasz-tétel!
\item Hozzunk létre táblázatot!
\item Szúrjunk be ábrát! Állítsuk be a szélességét fél
\verb+\textwidth+-re, 6~cm-re! Esetleg forgassuk el!
\item Hozzunk létre verbatim szövegrészt környezettel és \verb+\verb+
paranccsal!
\item Későbbre: Hozzunk létre egy \LaTeX-táblázatot psql paranccsal, és
 hozzuk be \verb+\input+ utasítással!
\begin{Verbatim}
psql -q -P format=latex -P border=3 -c "SELECT * FROM betetesek;"
\end{Verbatim}
 (Lásd még \verb!linux/postgresql/bank/kimutatas.sh! fájlt.)
\end{itemize}

\section{Megjegyzések, nem kötelező anyag}
\begin{itemize}
\item
\begin{Verbatim}
pygmentize -O full -o bashrc.tex -f tex -l bash ~/.bashrc
\end{Verbatim}

\item latex2html: Esetleg fordítsuk le a 
\verb!linux/latex/peldak/komplex/komplex.tex! fájlt weboldalakká. Segít
a \verb!Makefile!.
\end{itemize}

\section{A make használata}
A \texttt{make paranccsal} lehet több egymástól függő fordítást irányítani.
Eredetileg a C-forrásfájlok fordítására készült. Képes arra, hogy csak a
frissült C-forrásfájlokat fordítsa újra. Hasznos lehet
például \LaTeX-forrásfájlok fordításánál is.

A \texttt{make} parancs a \texttt{Makefile} vagy \texttt{makefile} fájlt
keresi az aktuális könyvtárban, és azalapján dolgozik.  A
\texttt{latex/peldak/komplex} könyvtárban található egy példa:
\begin{Verbatim}[label=\fbox{Makefile}, frame=single]
komplex.pdf: komplex.tex
	pdflatex komplex.tex
	pdflatex komplex.tex

html: komplex.tex
	latex2html komplex.tex

clean:
	rm *.log *.aux
\end{Verbatim}

Ebben az esetben, ha sima \texttt{make} parancsot adok ki, akkor a
legelső szakasz hajtódik végre, kétszer lefordítja a \LaTeX-fájlt
PDF-be, de csak akkor, ha a \texttt{komplex.tex} fájl újabb, mint a
\texttt{komplex.pdf},
vagy egyáltalán nincs \texttt{komplex.pdf}.
Ugyanez a helyzet, ha az alábbi első sort írtuk volna be:
\begin{Verbatim}
$ make komplex.pdf
$ make html
\end{Verbatim}
A második sor HTML-be fordítja a \LaTeX-fájlt tartalmát. A html nem egy
létező fájlnév (nem is jön létre), tehát bármikor hívva végrehajtódik az
utána található parancs.

Vimben kettőspont után írhatunk make parancsokat, mint shellben.
Kettőspont sem kell make elé.

A szintaktákánál fontos, hogy a parancsokat tartalmazó rész tabulátorral
legyen behúzva. Végrehajtáskor minden egyes parancssor külön shellben kerül
megnyitásra, tehát nincs mód a burokváltozók felülírására.

A make parancs lehetőséget ad arra is, hogy több gépen fordítsunk le egy
nagyobb -- sok C- vagy C++-kódot tartalmazó -- forrást. Ehhez a
\texttt{distcc} parancs szolgál.
Bővebb leírás erről és általában a make-ről az alábbi oldalon található
prezentációban:
\begin{Verbatim}
http://cern.ch/ahorvath/prez/make_rpm
\end{Verbatim}

\newpage
\section{Emberi nyelvek, karakterkódolások}

Házi feladat köv alkalomra: Ismerkedés a Pythonnal
\url{http://pythontutorial.pergamen.hu} honlap alapján. Átnézni az alábbi
fejezeteket: 3. Kötetlen bevezető... 5.  Adatstruktúrák

\subsection{Emberi nyelvek}

Otthon beállíthatjuk, hogy ne kelljen mindig külön
konzolt nyitni a rootnak:
\begin{itemize}
\item Ezegyszer utoljára jelentkezzünk be rootként.
\item Állítsuk be a sudo-t: \code{visudo} parancs
\item Másoljuk át a root-os sort (\code{yyp})
\item Az új sorban cseréljük a \code{root} szót \code{diak}-ra (\code{cw})
\item Lépjünk ki (\code{<Esc>:wq})
\end{itemize}
Ekkor mindenre hatalmunk lesz diak felhasználóként is, ha a parancs elé
biggyesztem a sudo parancsot. Szerveren nem illik ez a mindent
megengedő beállítás, ott a rendszergazda csak egy-egy részterületet
(webszerver kezelése, adatbázis mentése) szokott másra bízni.

Visszatérve normál felhasználói módba -- állítsuk be, milyen
nyelveket ismerjen a gép, legyenek ezek (magyar latin2-es és UTF-8:
hu\_HU, hu\_HU.UTF-8 német de\_DE, amerikai: en\_US):
\begin{Verbatim}
sudo dpkg-reconfigure locales
\end{Verbatim}
Miután végeztünk, nézzük meg a /etc/locale.gen tartalmát!

Amíg tovább dolgozunk, futtassuk le egy másik ablakban a \code{sudo
updatedb} parancsot.

Vannak olyan programok, amelyek a LANG változót nézik, mások a LANGUAGE
változót. Nézzük meg a tartalmukat: \code{export | grep LANG}.
Próbáljuk meg működnek-e (nem kell mindet kipróbálni):
\begin{Verbatim}
export LANG=de_DE
vimtutor (kilépés :q)
abiword&	(ha van ilyen)
export LANGUAGE=de_DE
vim   (majd kilépés)
gimp &
\end{Verbatim}
Miután kipróbáltuk, állítsuk mindkettőt vissza hu\_HU-ra.

A magyarítási fájlok egy .mo kiterjesztésű fájlban találhatóak. Amely a
szerkeszthető .po fájlokból készülnek. Töltsük le a \alapcim/vim
oldalról a vim.po fájlt (és a vimrc-t mentsük el ~/.vimrc néven, ha még
nem tettük). Később kell majd ennek a szülőkönyvtárából a galaktos\_utf-8
nevű fájl is.

Ha megnézzük a tartalmát, akkor msgid sorokban angol, msgstr sorokban
magyar szövegeket találunk. Ilyen -- de magyar szövegek nélküli -- fájl
hozható létre a Vim  és más programok forráskódjából, amelyet mindenféle
nyelvre le lehet fordítani. Az updatedb bizonyára lefutott, ez egy
adatbázist hozott létre a fájlok/könyvtárak neveiből, ami alapján
gyorsan kereshetünk fájlokat, keressük meg a \code{vim.mo} fájlok
helyét:
\begin{Verbatim}
locate vim.mo
\end{Verbatim}

Látszik, hogy még hu-s változat nincs. Fordítsuk le a gép által ismert
.mo fájllá a .po fájlunkat. (Ehhez a gettext csomag kell.) Másoljuk be
a megfelelő helyre. És indítsuk el magyar Vimünket.
\begin{Verbatim}
msgfmt -o vim.mo vim.po
sudo mkdir -p /usr/share/vim/vim63/lang/hu/LC_MESSAGES/
sudo cp vim.mo /usr/share/vim/vim63/lang/hu/LC_MESSAGES/
vim
\end{Verbatim}
(A mkdir a -p opcióval úgy csinál könyvtárat, hogy a még nem létező
szülőkönyvtárakat is létrehozza.)

Szerkeszthetjük is a fájlt szövegszerkesztővel, de hosszabb fájlok
esetén a kbabel, gtranslator vagy poedit használata könnyíti a
dolgunkat. Ezeken a gépeken a kbabel és a gtranslator van fent, csak ízelítésképpen
megpróbálkozhatunk a használatával. (\code{kbabel vim.po})

Csináljunk mi fordítható programot! Töltsük le a hello.py
fájlt a szokásos linalk könyvtárból, és kövessük az elején található
utasításokat.  (Egy Python-programot  magyarítunk, bővítjük a programot,
és ismét magyarítjuk.)

Miért érdemes ismerni ezeket a dogokat? Előfordulhat, hogy a cégnek,
ahol dolgozunk szüksége lenne egy olyan szabad szoftverként meglevő
programra, amely még nincs lefordítva, és a felhasználók nem tudnak
(kellő mértékben) angolul. Megkereshető általában az interneten egy CVS
vagy SVN archívumban a .pot fájl, amiből megcsinálhatjuk a mi .po
fájlunkat, lefordíthatjuk és már van működő magyar alkalmazásunk.
Természetesen örülnek a fejlesztők, ha visszaküldjük. A fordításnak van
egy webes változata, a launchpad.net, ahol elsősorban az Ubuntuban
található csomagok fordítása folyik, melyek közül a legtöbb nem csak az
Ubuntuban szerepel (pl. a Vim, a GIMP, és a Firefox M\$ Windows alatt is
megy).

Előfordulhat az is, hogy mi szeretnénk fordítható programot összeütni.

\newpage
\section{A fájlrendszer és gyakorlás}
%Lépjünk be a mail-szerverre!

\subsection{A fájlrendszer}
Közösen átnézni:

/etc: fstab group profile crontab apt/sources.list

/bin zgrep (\code{file /bin/*})

/usr: local share share/doc

/var: log (syslog, dmesg) www cache/apt/archives cache/locate

/tmp /home /root /sbin /dev /mnt (cdrom, floppy...) /lib

\section{Képek kezelése}
Képtípusok raszter/vektoros, tömörítés típusai.

\subsection{Képek átalakítása az Imagemagick programokkal}

Töltsük le a bash.tar.gz fájlt. Ennek kicsomagolásakor a bash/kepek
könyvtárban vannak képek.

Alakítsunk át egy képet másik formába:
\begin{Verbatim}
convert valami.png valami.jpg
\end{Verbatim}

Kérjünk adatokat a képeinkről:
\begin{Verbatim}
identify *
\end{Verbatim}

Tanulmányozzuk a \code{bash/kepek/kepatalakito} nevű szkripttel a
\code{convert} lehetőségeit.

Ehhez ismerni kell a következőket.
A \code{basename} utasítás használata.
Az argumentumlista kezelését tanulmányozzuk a \dir{bash/argumentum\_lista} fájljain.

Sok kép Exif dátumainak módosítása (ha nem jól járt a fényképezőgép
órája): \code{exiv2} parancs/csomag \code{-a} opció.

\subsection{Képek kezelése a gthumb programmal}
Nyissuk meg a gthumb programmal a képek könyvtárat. Adjunk megjegyzést a
képekhez (c billentyű). Listázzuk a rejtett állományokat a képek
könyvtárában (hova kerülnek a megjegyzések?).

Jelöljük ki a képek egy részét. (\Ctrl- ill. Shift-katt) és csináljuk egy
honlapra kitehető galériát ezekből: Eszközök $>$ Webalbum létrehozása

\newpage
\appendix

\section{Az AREK könyvtárában fellelhető irodalom}
A darabszámok az éppen bent találhatóak, azaz minimumadatok.
\subsection{UNIX általános/parancssor}

Bartók: UNIX felhasználói ismeretek 3db

Harley: UNIX bevezetés/haladó 3db

Kernighan -- Pike: A UNIX operációs rendszer, 1987, 6db, régi de alapos

Bagoly: UNIX alapismeretek 1995 5db

Büki: Héjprogramozás

\subsection{Linux}

Petersen: Linux, 1997, 2db, Parancssor mellett, (felhasználó/hálózat)adminisztráció, TeX

Gagné: Válts Linuxra! Búcsú a kékhaláltól, 2004, Hangsúlyosabb a grafikus felület, KDE

\subsection{Speciális területek}

Flickenger: Linux bevetés közben, 2003, Tippek

Butzen: Linux hálózatok

Kirch: Linux hálózati adminisztrátorok kézikönyve

Friedl: Reguláris kifejezések

Pere László: Linux felhasználói ismeretek I.

Pere László: Linux felhasználói ismeretek II, Adatkezelés (awk, bc, adatbázis)

Pere László: GNU/Linux rendszerek üzemeltetése I. és II.

Gagné: Linux rendszerfelügyelet

\section{Hasznos fogások}

Levél küldése bash-ből mail ($\in$ mailx).

%Az \code{ifconfig} parancs kimenetéből megtudhatom, hogyan
%jelentkezhetnek be a gépemre. 2. sorban inet addr: utáni négy számot
%írhatom a gép neve helyett. Pl:
%ssh diak@

\subsection{Érdemes lehet a .basrc-be tenni}

\hspace{1 ex}
\begin{Verbatim}[label=\fbox{~/.bashrc}, frame=single]
alias sshmaildiak1="ssh -p xx22 diak1@mail.roik.bmf.hu"
alias cx="chmod a+x"
alias ipythonlab='ipython -pylab'

export EDITOR="vim"  # A vim lesz az alapértelmezett szerkesztő
export LESSEDIT="%E ?lt+%lt. %f" # lessből a vim az adott sortól kezd
				 #szerkeszteni
export CDPATH=.:~   # cd parancsnál a homekönyvtárban is keres
\end{Verbatim}

\subsection{mc (Midnight Commander $\in$ mc)}

Beállítások $>$ alapbeállítások... $>$ Lynx-hez hasonló navigálást
érdemes beikszelni, akkor a jobb nyíllal be lehet menni a
könyvtárakba, bal nyilakkal szülőkönyvtárba $\Rightarrow$ nagyon gyors
navigálás.

Ha a belső szövegszerkesztő elől kivesszük az x-et, akkor az
alapértelmezett szövegszerkesztőnket kapjuk \verb+<F4>+-re. Fentebb van,
hogyan állíthatjuk ezt vim-re a .basrc-ben.

Grafikus felületen gyakran az F10 gomb másra foglalt. F9-cel kell
menübe menni Fájl $>$ Kilépés-sel lehet kilépni. Fájlokból kilépés két
\Esc-el.

Másik mód a GNOME-terminál esetén annak beállításával: Szerkesztés $>$
Gyorsbillentyűk: Menüelérés billentyűjének (F10) tiltása

\subsection{Pár dolog, amit együtt érdemes megjegyezni}

\noindent
\begin{tabularx}{\textwidth}{|l|X|X|}
\hline
/ & keresés &Firefox, Vim, man, less, info\\
\hline
n és N& adott irányban illetve ellenkező irányban keresi a következő
találatot&Vim, man, less\\
\hline
q& kilépés& man, less, mutt, info esetén\\
\hline
:q& kilépés& Vimből, ha nem volt módosítás\\
\hline
:q!& kilépés& Vimből, ha nem mentem a  módosításokat\\
\hline
\Ctrl$+$D& kilépés& shellből (ssh-t ill. terminálablakot bezárja); szövegbeíráskor  egy
szűrő (pl. cat, mail, grep) után, ipython, python shell\\
\hline
$<$F11$>$& teljes képernyő& GNOME terminál; Firefox; evince\\
\hline
\Ctrl$+<$PageUp$>$\quad \Ctrl$+<$PageDown$>$& váltás a lapok között
egyik ill. másik irányban&
GNOME terminál; Firefox\\
\hline
\Alt$+$1 \dots \Alt$+$9& adott sorszámú lapra ugrik& GNOME terminál; Firefox\\
\hline
\end{tabularx}

\subsection{GNOME, KDE és LXDE együtt Ubuntun}
Akár melyik változatot telepítettük az
\code{ubuntu-desktop},
\code{kubuntu-desktop},
\code{lubuntu-desktop} csomag telepítésével megkapjuk az adott
desktop környezetet.

\subsection{\code{sudo} Debianon is}
Otthon beállíthatjuk Debian alatt is, hogy ne kelljen mindig külön
konzolt nyitni a rootnak, hanem sudo-val dolgozhassunk:
\begin{itemize}
\item Ez egyszer utoljára jelentkezzünk be rootként.
\item Állítsuk be a sudo-t: \code{visudo} parancs
\item Másoljuk át a root-os sort (\code{yyp})
\item Az új sorban cseréljük a \code{root} szót \code{diak}-ra (\code{cw})
\item Lépjünk ki (\code{\Esc:wq})
\end{itemize}
Ekkor mindenre hatalmunk lesz diak felhasználóként is, ha a parancs elé
biggyesztem a sudo parancsot. Szerveren nem illik ez a mindent
megengedő beállítás, ott a rendszergazda csak egy-egy részterületet
(webszerver kezelése, adatbázis mentése) szokott másra bízni.

\newpage
\tableofcontents
\end{document}
